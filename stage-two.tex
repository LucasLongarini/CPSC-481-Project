\documentclass{article}
\usepackage[margin=1in]{geometry}
\usepackage[utf8]{inputenc}

\title{Team H Stage Two}
\author{Lucas Longarini, Abel Tekeste, Josie Khampheng, \\
Joel Poirier, Maiah Rutledge}
\date{October 18, 2020}

\usepackage{natbib}
\usepackage{graphicx}
\usepackage{enumerate}
\usepackage[toc,page]{appendix}
\usepackage{url}
\usepackage{hyperref}

\renewcommand\familydefault{\sfdefault}
\begin{document}

\maketitle

% Introduction
\section{Introduction}
\paragraph{} The project idea our team decided on is to design a Roommate Review app, wherein
users can search for roommates and landlords based on their profiles on the app, which contain 
information about these users, personality types, personal information, and reviews from other 
users. The system will be used as a mobile phone application and can be downloaded by anyone, 
though its intended use is by individuals who are renters or landlords. 
\medskip
\paragraph{} Overall, the system solves the problem of roommates and landlords often not knowing 
much about each other prior to consenting to live together, and it solves this by helping all 
stakeholders in a renter-landlord situation to have access to a much more detailed and diverse 
source of information.
\medskip
\newline

\section{Users \& Stakeholders}
\begin{description}
    \item[Renters (User \& Stakeholder):]
    \newline \,
    \begin{itemize}
        \item Renters who are looking to gain information about potential roommates/landlords. 
        These renters would essentially want to find the best possible match for a roommate or  
        landlord by using the app and seeing how their personal preferences and personality traits
        compare to those of other people on the app. This would be facilitated by the personality 
        test that each user would have to use when starting the app. 
        \item For renters that are already living with a roommate or have recently moved out and 
        would like to rate their experience with them, this app can also be used to leave reviews 
        of these roommates once the rental term ends so that they may advise future roommates of 
        the person of what their experience was like, to overall increase the available 
        information for these roommates. 
        \item Renters must be able to use technology at least to the point where they can use the 
        app proficiently. There is no need for them to have background knowledge on the app, but 
        it would help to have some experience with it. 
        \item Renters will complete a personality survey when they sign up for the app so that 
        there is at least a little bit of information about them. 
    \end{itemize}
    
    \item[Landlords (User \& Stakeholder):]
    \newline \,
    \begin{itemize}
        \item Landlords who are looking to gain information about potential renters.  These 
        landlords would essentially want to find the best possible match for a renter by using the 
        app and seeing how each of their personal preferences and personality traits compare to 
        those of other people on the app. This would be facilitated by the personality test that 
        each user would have to use when starting the app. 
        \item Landlords who would like to detail their experiences with past or current renters. 
        The landlords would want to detail their experience with their current or past renters and 
        also give them a rating. The app facilitates this by allowing them to fill out a review 
        survey that would be visible to other people on their renter’s profile page.
        \item Landlords must be able to use technology at least to the point where they can use the
        app proficiently. It is not mandatory, but the landlords should understand the features of 
        the app and have background knowledge on the app in order to help renters with any problems
        they might encounter.
        \item Landlords must complete a personality survey when they sign up for the app. 
    \end{itemize}
    
    \item[Design Team (Stakeholder):]
    \newline \,
    \begin{itemize}
        \item Responsible for designing the system, consulting users, and reporting to supervisors.
    \end{itemize}
    
    \item[Supervisors (Stakeholder):]
    \newline \,
    \begin{itemize}
        \item Responsible for ensuring the team stays on track and meets objectives through weekly 
        meetings and periodic grade-based feedback.
    \end{itemize}
    
\end{description}

\\~\\
% User Research Methods
\section{User Research Methods}
\subsection{Surveys \& Questionnaires}
\begin{description}
    \item[Justification:]
        \newline \,
        \begin{itemize}
            \item Surveys are a quick way to obtain information from a large number of users  because 
            they are easily accessible. It allows us to gain insight into our users' problems, and 
            find suggestions for features we can implement. We also want their opinion on existing 
            apps that solve similar issues.
        \end{itemize}
    \item [Summary:]
        \newline \,
        \begin{itemize}
            \item  Most participants had their friends or family members as roommates. The vast 
            majority of users valued someone who is clean and quiet. Many of the users also enjoyed 
            having someone with a similar or compatible personality. Dirty, loud, and rude were the 
            most common deal breaker for users. Not being respectful of personal space or privacy was 
            another common deal breaker. When meeting a roommate for the first time, approximately 
            82\% of participants preferred meeting in person.
            \item Users had mixed opinions when asked if they would use an app similar to Tinder to 
            find roommates. Users were more likely to use an app similar to Google Reviews to find 
            roommates. Most users were concerned with bots and fake reviews in these apps.
            \item When users were asked if they would take a personality test to help match them with 
            roommates, approximately 82\% said yes. Profile page, reviews page and personality test 
            were the features that users wanted most in our app. They wanted the ability to find and 
            review landlords as well.
        
        \end{itemize}
    \item [What went well:]
    \newline \,
        \begin{itemize}
            \item There were a fair number of participants, which gave us a good amount of information.
            \item The survey allowed us to better understand our users' problems.
            \item Gave us possible features for our design, and problems with pre-existing designs.
        \end{itemize}
    \item [What went poorly:]
    \newline \,
        \begin{itemize}
            \item Could not find as many users to take the survey as we had hoped, since most people our age 
            are still living with their parents.
        \end{itemize}
    \item [What would you do differently:]
    \newline \,
        \begin{itemize}
            \item Find more people to take the survey, and make the survey more general so people who are 
            not currently renting can take it.
        \end{itemize}
\end{description}
\begin{center}
    \includegraphics[scale=0.09]{Surveys.jpg}
\end{center}
\begin{center}
    \textit{Team member viewing survey results in a socially distanced environment.}
\end{center}

\subsection{Scenarios}
\begin{description}
    \item[Justification:]
    \newline \,
    \begin{itemize}
        \item  Scenarios are a great way to think about varied specific user experiences. They can 
        help designers to get a better perspective of what the user might experience using the 
        system and helps them to think about how to address and accommodate for different user types 
        and experiences among them.
    \end{itemize}
    
    \item[Summary:]
    \newline \,
    \begin{itemize}
        \item Thought about how app usage would look, such as from the perspective of a new renter 
        looking for roommates and a landlord, a current renter looking to review their roommates and 
        landlord, and a landlord looking to choose a new tenant.
        \item Quickly found issues with our current vision of the app that we had not previously 
        considered. An example of this was considering if one roommate wanted to leave a bad review 
        of another, if they were in the middle of their lease it may create tensions and result in a 
        negative experience. As a response to this Scenario, we decided that reviews for roommates 
        should be left after the lease has ended, to avoid any conflicts.
        \item As a response to this Scenario, we decided that reviews for roommates should be left 
        after the lease has ended, to avoid any conflicts or biases.
        \item Iterated on our app design to meet the needs of issues found through our Scenarios.
    \end{itemize}
    
    \item[What went well:]
    \newline \,
    \begin{itemize}
        \item Scenarios made it easy to view the app from the perspective of a user, rather than just 
        thinking about it as designers. This helped us to quickly and easily think about the current 
        pros and cons of our design to address issues within the design.
        \item Allowed the team to empathize with the users which helped to inspire design changes in 
        a much deeper way than would be done acting just on our existing ideas.
    \end{itemize}
    
    \item[What went poorly:]
    \newline \,
    \begin{itemize}
        \item Scenarios were limited by what we could think of in a brainstorming session. During the 
        real usage of the app, there would likely be edge cases not previously considered.
        \item Scenarios were limited due to user types of a renter and a landlord. 
    \end{itemize}
    
    \item[What would you do differently:]
    \newline \,
    \begin{itemize}
        \item To address limited scenarios, it would have been good to consider more how different 
        demographics within those categories may use the app.
    \end{itemize}
\end{description}
\\~\\
\begin{center}
    \includegraphics[scale=0.3]{Scenarios.jpg}
\end{center}
\begin{center}
    \textit{The team members in a socially distanced meet-up discussing various Scenarios}
\end{center}

\subsection{Activity Analysis}
\begin{description}
    \item[Justification:]
    \newline \,
    \begin{itemize}
        \item Activity analysis is a great way to analyze the activities involved in learning about a 
        roommate/landlord. This method helped us analyze all the activities we could think of, which 
        brought to light some unanticipated concerns and challenges. This method also helped us 
        identify stakeholders and users to interview and tied nicely into the two previous methods.
    \end{itemize}
    
    \item[Summary:]
    \newline \,
    \begin{itemize}
        \item We started by thinking of ways people look for roommates, landlords, and tenants and the 
        activities involved. This lead us to think about which attributes people want to learn about 
        potential roommates and ways in which they discover this.
        \item Brainstorming these ideas helped us discover areas that work and areas that could be 
        improved. This gave us insight into potential solutions our app could provide for users.
    \end{itemize}
    
    \item[What went well:]
    \newline \,
    \begin{itemize}
        \item Brainstorming activities that potential users of our app go through today, helped think 
        of possible solutions and “nice-to-haves”. This method brought to light a couple of concerns 
        and needs that were unanticipated initially.
        \item This also allowed the team to empathize with users of the app and think of more creative 
        and deeper ideas that had not been discovered yet.    
    \end{itemize}
    
    \item[What went poorly:]
    \newline \,
    \begin{itemize}
        \item No team members involved in this method had the actual experience of looking for a 
        roommate/landlord. This made it more challenging to think of every activity involved since we 
        didn't have the same experience as our users.
    \end{itemize}
    
    \item[What would you do differently:]
    \newline \,
    \begin{itemize}
        \item It would have been better to use real users/stakeholders to help with this research 
        method. Actual users would have been able to give us a better picture and reduce the risk of 
        any false assumptions we may have made. 
        \item Completing this researched method before the Surveys & Questionnaires method would help 
        incorporate the results we discovered into the surveys
    \end{itemize}
\end{description}
\\~\\
\begin{center}
    \includegraphics[scale=0.3]{ActivityAnalysis.jpg}
\end{center}
\begin{center}
    \textit{Re-enactment of team member jotting down notes.}
\end{center}

% User Task Descriptions
\newpage
\section{User Task Descriptions}
\begin{description}
    \item[Must be included:]
        \newline \,
        \begin{itemize}
            \item Roommates/landlords are able to rate and review each other
            \item Users can search for potential roommates and landlords
            \item Users have a public personal profile containing information about themselves
            \item Users can take a personality test to help match them to similar roommates
        \end{itemize}
    \item[Important:]
        \newline \,
        \begin{itemize}
            \item Users can message potential roommates
            \item Users can create and join groups which contain members of the same household
            \item Link landlords/roommates with a property address
            \item Users can filter potential roommates and landlords by specific preferences or rating
        \end{itemize}
    \item[Could be included:]
        \newline \,
        \begin{itemize}
            \item Each user has a friends list and can add other users as friends
            \item Roommates can chat with landlords through the app
            \item App can recommend potential matches automatically through similarities between users
        \end{itemize}
\end{description}

\\~\\
Online repository: \url{https://github.com/LucasLongarini/CPSC-481-Project}
\newline
Online portfolio: \url{https://lucaslongarini.github.io/CPSC-481-Project/index.html}

% APPENDICES
\newpage
\appendix
\begin{appendices}
\section{Surveys \& Questionnaires}
Survey can be viewed at \url{https://forms.gle/VhGEReZ57i6Y89E59}
\begin{enumerate}
    \item What is your relationship with past/current roommates before you moved in?
    \begin{itemize}
        \item Family: 3 (27.2\%)
        \item Friends: 6 (54.4\%)
        \item Acquaintance: 2 (18.2\%)
        \item Stranger: 2 (18.2\%)
        \item Significant other (18.2\%)
    \end{itemize}

    \medskip
    \item{What do you look for in a roommate?}
    \begin{itemize}
        \item Clean, Friendly, Gender isn't relevant but it's nice being roommates with other queer 
        people, similarish age/place in life, open minded to weird art projects/music/etc.
        \item Someone who is clean, keeps the place quiet, friendly and respectful.
        \item Clean and i should get along with them
        \item attitude
        \item Friendly, quiet and clean 
        \item Clean, quiet
        \item Willingness to compromise/Some cleanliness 
        \item Clean, responsible, quiet at night
        \item Clean and does not make a lot of noise. 
        \item Similar schedules, respect privacy, someone I'm comfortable with
        \item Cleanliness, willingness to split chores
    \end{itemize}
    
    \medskip
    \item{What are your deal breakers in a roommate?}
        \begin{itemize}
            \item If they absolutely ignore me as a human, they leave messes for a long time, party 
            animals who are too loud and not respectful, racist/homophobic/rude
            \item Messy, Loud, Rude, Respects personal space
            \item Messy
            \item personality 
            \item Messy 
            \item Dirty, loud (especially late at night), abrasive personality, likes small talk
            \item Not respecting my property
            \item Dirty, rude, loud
            \item People that don't care about personal space. 
            \item Cross boundaries, heavy smokers/drinkers, keep me up at night
            \item Dirty, lazy
        \end{itemize}
    
    \newpage
    \item{When meeting a potential roommate, how do you prefer to communicate?}
    \begin{itemize}
        \item In person: 9 (81.8\%)
        \item Text: 1 (9.1\%)
        \item Video Call: 1 (9.1\%)
    \end{itemize}
    
    \medskip
    \item{Would you take a personality test to help match you with roommates?}
    \begin{itemize}
        \item Yes: 9 (81.8\%)
        \item No: 1 (9.1\%)
        \item Prefer to personally know my roommates prior to moving in: 1 (9.1\%)
    \end{itemize}
    
    \medskip
    \item{Consider dating apps like Tinder, Bumble, etc. How likely are you to use a similar app to 
    find roommates?}
    \begin{itemize}
        \item 1: 1 (9.1\%)
        \item 2: 3 (27.3\%)
        \item 3: 2 (18.2\%)
        \item 4: 3 (27.3\%)
        \item 5: 2 (18.2\%)
    \end{itemize}
    
    \medskip
    \item{Consider review apps, like Google Reviews, Yelp, etc. How likely are you to use a similar app 
    to review roommates?}
    \begin{itemize}
        \item 1: 1 (9.1\%)
        \item 2: 0 (0\%)
        \item 3: 4 (36.4\%)
        \item 4: 4 (36.4\%)
        \item 5: 2 (18.2\%)
    \end{itemize}
    
    \medskip
    \item{What do you like or dislike about these apps?}
    \begin{itemize}
        \item Google reviews can be biased based on peoples lived experiances so I don't hold too much 
        stock. Dating app type could work, very niche but could work. I'd be worried it would be 
        gamified for finding someone.
        \item Apps like tinder, bumble, etc.. are simple to use and easy. Review apps are a bit more 
        complicated with more information and usually a less pleasing UI
        \item Idk
        \item bots fake profiles
        \item An app would be easy to talk to and meet people. Google reviews seem like mostly people 
        complaining and fake reviews
        \item Prevalence of bots
        \item Dating apps let you make quick matches, so you can quickly find roommates. Review pages 
        gives you an idea of what the person is like, but can be filled with fake reviews.
        \item Reviews might be fake sometimes. 
        \item I like that I can see the feedback of other users to give a general impression. I dislike 
        that sometimes people can have different perspectives of the criteria for what is good and what 
        is bad, so may rate more negatively/positively than I might.
        \item Examples like tinder would be too impersonal
    \end{itemize}
    
    \medskip
    \item{Would you use this app to find/review landlords?}
    \begin{itemize}
        \item 1: 1 (9.1\%)
        \item 2: 0 (0\%)
        \item 3: 0 (0\%)
        \item 4: 7 (63.6\%)
        \item 5: 3 (27.3\%)
    \end{itemize}
    
    \medskip
    \item{What features would you like to see in this app?}
    \begin{itemize}
        \item Friends list: 1 (16.7\%)
        \item Profile page: 6 (100\%)
        \item Personality test: 5 (83.3\%)
        \item Reviews page: 6 (100\%)
        \item Household groups: 3 (50\%)
        \item Messaging: 4 (66.7\%)
    \end{itemize}
\end{enumerate}

\newpage
\section{Scenarios}
\begin{itemize}
    \item Roommate searching for other roommates (join/find)

    \item Landlord looking for tenants
   
    \item Roommate(s) looking for landlord
    
    \item People reviewing you might not have been in same household and just randomly found you     
    \newline $\rightarrow{}$ Need some verification between roommates
        \newline $\rightarrow{}$ Rating only happens if all parties accept that they are in the same 
        household
        \newline $\rightarrow{}$Also verification should happen at start of moving in so that no bad 
        blood prevents any user from verifying each other
    
    \item Leaving a bad rating for a roommate
    \newline $\rightarrow{}$ Should happen after the term has ended to avoid tensions between roommates
    
    \item A user with bad reviews wants to delete the app and make a new account
    \newline $\rightarrow{}$ Name and phone number should be tied to an account so that users can’t just 
    keep making new accounts
    
    \item Users might get their friends to rate them positively
    \newline $\rightarrow{}$ Hard to avoid this, even happens on Google Reviews
    
    \item Landlord may or may not be on app
    \newline $\rightarrow{}$ Not necessary to confirm household for ratings, but if the landlord is on 
    the app you can create house groups and messaging between users in the group
\end{itemize}


\newpage
\section{Activity Analysis}
\begin{itemize}
    \item When searching for a roommate, other someone will:
        \begin{itemize}
            \item Post ads or ask friends/family to find someone
            \begin{itemize}
                \item Solution: Would be nice to have a place to look (like a job board)
            \end{itemize}
        \end{itemize}
    \item To discover information about a potential roommate, someone will:
        \begin{itemize}
                \item Talk/meet the potential roommate
                \item Ask others about that roommates personality
                \begin{itemize}
                    \item Solution: Would be nice to see others reviews/comments about the potential 
                    roommate.
                \end{itemize}
        \end{itemize}
    \item Types of information someone would want to learn about a roommate:
        \begin{itemize}
            \item Personality
                \begin{itemize}
                    \item Solution: Personality tests could help
                \end{itemize}
            \item Attributes
                \begin{itemize}
                    \item Solution: Clean/Rude rating metrics
                    \item Solution: Clean/Messy rating metrics
                    \item Solution: Loud/Quiet rating metrics
                \end{itemize}
        \end{itemize}
    \item When searching for a landlord, someone will:
        \begin{itemize}
            \item Usually just contact the number of the property they are interested in
        \end{itemize}
    \item To discover information about a landlord, someone will:
        \begin{itemize}
            \item Find out by experience, meet them, or get information from a past tenet.
            \begin{itemize}
                \item Solution: Would be nice to see others reviews/comments about the potential landlord
            \end{itemize}
        \end{itemize}
    \item Type of information someone would like to learn about a landlord:
        \begin{itemize}
            \item Attributes
            \begin{itemize}
                \item Solution: Friendly/Strict rating metrics
                \item Solution: Communication rating metrics
            \end{itemize}
        \end{itemize}
    \item Landlords looking for roommates will:
        \begin{itemize}
            \item Usually just reply to applications
        \end{itemize}
    \item How do landlords get information about potential tenants?
        \begin{itemize}
            \item Go off past experience/references
            \begin{itemize}
                \item Learn by experience
                \item Solution: Would be nice to see the same reviews that roommates can see about people.
            \end{itemize}
        \end{itemize}
\end{itemize}

\end{appendices}
\end{document}
