\documentclass{article}
\usepackage[utf8]{inputenc}
\usepackage{hyperref}

\title{Team H Project Proposals}
\author{Abel Tekeste, Lucas Longarini, Josie Khampheng,
\\Joel Poirier, Maiah Rutledge}
\date{October 2, 2020}

\begin{document}

\maketitle

\section{Note-Taking Application}
\paragraph{}
The problem we are trying to solve is the lack of easy to use, well-designed, and well-organized note taking apps. This app would provide a friendly and intuitive user interface that users would ideally pick up easily.
The core importance of this app is with its user centered design. Most note taking apps take time to learn and use. We aim to solve the learning curve associated with most note taking apps so the user has a better experience.
We would solve this problem by thinking of the user first. Designs would go through many iterations and usability tests to understand what works and what doesn’t.
We would find where users' frustrations and confusions lie with existing apps and attempt to solve this. This service would be a computer software, mainly for tablets and touch screens.
One of the main features would be a pen tool that users can use to write hand notes from their device.

\section{Roommate Reviews}
\paragraph{}
When moving into a new place of residence, it is difficult to tell if you will get along with your roommates.  You often do not see your roommates' true personality until you get to know them for a period of time.
This app aims to match people with compatible roommates based on personality type and preferences, such as waking hours, drinking habits, and smoking habits. Users will also be able to review one another.
This app is important because it allows users to avoid potential conflicts. This is especially important when signing long-term leases, where personality conflicts can make for an unlivable situation.
We want the app to be easy to use with a low learning curve, because we aim for it to be used by a wide variety of people. We also need to ensure that the app covers a variety of personality types and preferences.
We will solve this problem by gathering information from users, such as aspects of their personality and what they like/dislike in a roommate. 
This service will be a mobile app to make use of notifications and location tracking features of mobile devices.


\section{Movie Theatre Ticket Booking}
\paragraph{}
The problem we are trying to solve is the lack of online/mobile ticket booking services that are available for movies.
The service we are going to be providing is the Movie Theatre Ticket Booking system that will allow you to choose a theatre and book a movie ticket there.
Franchises like Cineplex have already implemented the feature to pre-book a ticket with the help of their app. However, there is a vast majority of theatres across the country that don’t allow you to do this.
It is important that we offer this service as it will drastically decrease the amount of time people stand in line for tickets at whichever theater they go to and allow them the chance to just walk in to their movie hassle-free.
We will solve this problem by introducing a simple and easy to use web and mobile application that would solve the frustrations of being in line at a movie theatre and deal with sold out shows.
Our main focus with this app is to save the user time by allowing them to quickly and efficiently purchase and manage their tickets. One main feature to help with this issue would be graphical seat reservation.
We would also need to add a lot of security functionality as there will be payments happening through the application. Initially,  we can start off with a website and then expand into a mobile app for tablets and mobile devices.

\\~\\

\begin{flushleft}  
Portfolio:
\\ \url{https://lucaslongarini.github.io/CPSC-481-Project/index.html}
\\~\\
Repository:
\\ \url{https://github.com/LucasLongarini/CPSC-481-Project}
\end{flushleft}  

\end{document}
